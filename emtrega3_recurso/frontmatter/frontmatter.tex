% we include the glossary here (frontmatter is included with \input, so this command is as if it was in main.tex)
%%All acronyms must be written in this file.

% Core Concepts
\newacronym{AI}{AI}{Artificial Intelligence}
\newacronym{LLM}{LLM}{Large Language Model}
\newacronym{MAS}{MAS}{Multi-Agent System}
\newacronym{NLP}{NLP}{Natural Language Processing}
\newacronym{ML}{ML}{Machine Learning}

% Software Engineering
\newacronym{SE}{SE}{Software Engineering}
\newacronym{SDLC}{SDLC}{Software Development Life Cycle}
\newacronym{CI}{CI}{Continuous Integration}
\newacronym{CD}{CD}{Continuous Delivery}
\newacronym{API}{API}{Application Programming Interface}
\newacronym{IDE}{IDE}{Integrated Development Environment}
\newacronym{TDD}{TDD}{Test-Driven Development}

% Testing Related
\newacronym{SUT}{SUT}{System Under Test}
\newacronym{SBST}{SBST}{Search-Based Software Testing}
\newacronym{AST}{AST}{Abstract Syntax Tree}
\newacronym{CRUD}{CRUD}{Create, Read, Update, Delete}
\newacronym{HAR}{HAR}{HTTP Archive}
\newacronym{REST}{REST}{Representational State Transfer}

% Agent and Architecture
\newacronym{ACI}{ACI}{Agent-Computer Interface}
\newacronym{RAG}{RAG}{Retrieval-Augmented Generation}
\newacronym{HITL}{HITL}{Human-in-the-Loop}
\newacronym{CoT}{CoT}{Chain-of-Thought}

% Security and Privacy
\newacronym{GDPR}{GDPR}{General Data Protection Regulation}
\newacronym{PII}{PII}{Personally Identifiable Information}
\newacronym{DPIA}{DPIA}{Data Protection Impact Assessment}
\newacronym{DPO}{DPO}{Data Protection Officer}

% Research Methodology
\newacronym{SLR}{SLR}{Systematic Literature Review}
\newacronym{RQ}{RQ}{Research Question}
\newacronym{PRISMA}{PRISMA}{Preferred Reporting Items for Systematic Reviews and Meta-Analyses}

% Models and Frameworks
\newacronym{GPT}{GPT}{Generative Pre-trained Transformer}
\newacronym{LLaMA}{LLaMA}{Large Language Model Meta AI}

 % the command makenoidxglossaries requires that the glossary entries must be defined in the preamble (to be compatible with overleaf)

\frontmatter % Use roman page numbering style (i, ii, iii, iv...) for the pre-content pages

\pagestyle{plain} % Default to the plain heading style until the thesis style is called for the body content

%----------------------------------------------------------------------------------------
%	TITLE PAGE
%----------------------------------------------------------------------------------------

\maketitlepage

%----------------------------------------------------------------------------------------
%	STATEMENT OF INTEGRITY
%----------------------------------------------------------------------------------------

\begin{soi}

I hereby declare that I have conducted this academic work with integrity.

I have not plagiarised or applied any form of undue use of information or falsification of results along the process leading to its elaboration.

Therefore, the work presented in this document is original and was authored by me, having not been previously used for any other purpose. Artificial Intelligence tools (specifically Large Language Models) were used as assistive tools for literature search, text refinement, and code development assistance. All AI-generated content was reviewed, validated, and integrated by the author with full responsibility for the final output.

I further declare that I have fully acknowledged the Code of Ethical Conduct of P.PORTO.

ISEP, Porto, \today

\end{soi}


%----------------------------------------------------------------------------------------
%	DEDICATION  (optional)
%----------------------------------------------------------------------------------------
%
%\dedicatory{For/Dedicated to/To my\ldots}
\begin{dedicatory}
To my family, for their unwavering support and patience throughout this journey.

To all software engineers striving to build more reliable and secure systems.
\end{dedicatory}

%----------------------------------------------------------------------------------------
%	ABSTRACT PAGE
%----------------------------------------------------------------------------------------

\begin{abstract}

Software testing remains a critical yet resource-intensive activity in modern software development. While Large Language Models (LLMs) have demonstrated promising capabilities for automated test generation, their deployment in enterprise environments raises significant security and privacy concerns. This thesis investigates the application of Multi-Agent Systems (MAS) powered by LLMs for automated software testing, with particular emphasis on security and privacy protection.

Following the PRISMA methodology, we conduct a systematic literature review addressing six research questions spanning security risks, mitigation strategies, testing effectiveness, architectural patterns, practical deployment challenges, and LLM configuration impacts. Based on the findings, we propose a secure-by-design reference architecture incorporating defense-in-depth security controls, including sandboxed execution environments, PII scrubbing mechanisms, Agent-Computer Interface hardening, and comprehensive audit logging.

A prototype implementation demonstrates the feasibility of the proposed architecture through six specialized agents: planning, code analysis, test generation, execution, validation, and security. Experimental evaluation assesses testing effectiveness against established benchmarks, security control efficacy against simulated attacks, and cost-performance trade-offs.

The thesis contributes a taxonomy of MAS testing architectures, a comprehensive security analysis, practical deployment guidelines aligned with GDPR and EU AI Act requirements, and empirical evidence supporting the viability of privacy-preserving autonomous testing systems.

\end{abstract}

\begin{abstractotherlanguage}

Os testes de software permanecem uma atividade crítica mas intensiva em recursos no desenvolvimento moderno de software. Embora os Modelos de Linguagem de Grande Escala (LLMs) tenham demonstrado capacidades promissoras para geração automatizada de testes, a sua implementação em ambientes empresariais levanta preocupações significativas de segurança e privacidade. Esta tese investiga a aplicação de Sistemas Multi-Agente (MAS) alimentados por LLMs para testes automatizados de software, com ênfase particular na proteção de segurança e privacidade.

Seguindo a metodologia PRISMA, realizamos uma revisão sistemática da literatura abordando seis questões de investigação que abrangem riscos de segurança, estratégias de mitigação, eficácia dos testes, padrões arquiteturais, desafios de implementação prática e impactos da configuração de LLMs. Com base nos resultados, propomos uma arquitetura de referência segura por design, incorporando controlos de segurança em profundidade, incluindo ambientes de execução isolados, mecanismos de remoção de dados pessoais, hardening da Interface Agente-Computador e logging de auditoria abrangente.

Uma implementação de protótipo demonstra a viabilidade da arquitetura proposta através de seis agentes especializados: planeamento, análise de código, geração de testes, execução, validação e segurança. A avaliação experimental analisa a eficácia dos testes contra benchmarks estabelecidos, a eficácia dos controlos de segurança contra ataques simulados e os trade-offs custo-desempenho.

A tese contribui com uma taxonomia de arquiteturas de testes MAS, uma análise de segurança abrangente, diretrizes práticas de implementação alinhadas com os requisitos do RGPD e do Regulamento Europeu de IA, e evidências empíricas que suportam a viabilidade de sistemas de testes autónomos que preservam a privacidade.

\end{abstractotherlanguage}

%----------------------------------------------------------------------------------------
%	ACKNOWLEDGEMENTS (optional)
%----------------------------------------------------------------------------------------

\begin{acknowledgements}

I would like to express my sincere gratitude to my supervisor, Professor Constantino Martins, for his guidance, expertise, and continuous support throughout this research. His insights into artificial intelligence and software engineering have been invaluable in shaping this work.

I am grateful to the faculty and staff of the Department of Computer Engineering at ISEP for providing an excellent academic environment and the resources necessary to conduct this research.

I would also like to thank my colleagues in the MEIA program for the stimulating discussions and collaborative spirit that enriched my learning experience.

Finally, I extend my deepest appreciation to my family and friends for their understanding, encouragement, and support during the demanding periods of this thesis work.

\end{acknowledgements}

%----------------------------------------------------------------------------------------
%	LIST OF CONTENTS/FIGURES/TABLES PAGES
%----------------------------------------------------------------------------------------

\tableofcontents % Prints the main table of contents

\listoffigures % Prints the list of figures

\listoftables % Prints the list of tables

\iflanguage{portuguese}{
\renewcommand{\listalgorithmname}{Lista de Algor\'itmos}
}
\listofalgorithms % Prints the list of algorithms
\addchaptertocentry{\listalgorithmname}


\renewcommand{\lstlistlistingname}{List of Source Code}
\iflanguage{portuguese}{
\renewcommand{\lstlistlistingname}{Lista de C\'odigo}
}
\lstlistoflistings % Prints the list of listings (programming language source code)
\addchaptertocentry{\lstlistlistingname}


%----------------------------------------------------------------------------------------
%	ABBREVIATIONS
%----------------------------------------------------------------------------------------
%\begin{abbreviations}{ll} % Include a list of abbreviations (a table of two columns)
%%\textbf{LAH} & \textbf{L}ist \textbf{A}bbreviations \textbf{H}ere\\
%%\textbf{WSF} & \textbf{W}hat (it) \textbf{S}tands \textbf{F}or\\
%\end{abbreviations}

%----------------------------------------------------------------------------------------
%	SYMBOLS
%----------------------------------------------------------------------------------------

\begin{symbols}{lll} % Include a list of Symbols (a three column table)

$LC$ & Line Coverage & \si{\percent} \\
$BC$ & Branch Coverage & \si{\percent} \\
$MS$ & Mutation Score & \si{\percent} \\
$CR$ & Compilation Rate & \si{\percent} \\
$ER$ & Execution Rate & \si{\percent} \\
$PR$ & Pass Rate & \si{\percent} \\
$AD$ & Assertion Density & assertions/test \\

\addlinespace % Gap to separate evaluation metrics from other symbols

$k$ & Number of generation attempts (pass@k) & -- \\
$n$ & Total number of samples & -- \\
$c$ & Number of correct samples & -- \\

\end{symbols}



%----------------------------------------------------------------------------------------
%	ACRONYMS
%----------------------------------------------------------------------------------------

\newcommand{\listacronymname}{List of Acronyms}
\iflanguage{portuguese}{
\renewcommand{\listacronymname}{Lista de Acr\'onimos}
}

%Use GLS
\glsresetall

%\printglossary[title=\listacronymname,type=\acronymtype,style=long]
\printnoidxglossary[title=\listacronymname,type=\acronymtype,style=long] % command compatible with overleaf

%----------------------------------------------------------------------------------------
%	DONE
%----------------------------------------------------------------------------------------

\mainmatter % Begin numeric (1,2,3...) page numbering
\pagestyle{thesis} % Return the page headers back to the "thesis" style
